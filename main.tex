\documentclass{report}
\usepackage[
  inner	=	3.0cm, % Margen interior
  outer	=	2.5cm, % Margen exterior
  top	=	2.5cm, % Margen superior
  bottom=	2.5cm, % Margen inferior
  includeheadfoot, % Incluye cabecera y pie de página en los márgenes
]{geometry}
% Valor de interlineado
\renewcommand{\baselinestretch}{1.0} % 1 línea de interlineado

% Paquetes imprescindibles para la portada
\usepackage{fontspec} % Para poder importar las fuentes
\usepackage{textpos} % Para posicionar bloques de texto
\usepackage{graphicx} % Para cargar imágenes
\graphicspath{{./Imagenes}}
\usepackage{setspace} % Para modificar los espacios arriba y abajo
%Otros paquetes
\usepackage[spanish]{babel}
\usepackage{fancyhdr}
\usepackage{hyperref} 
\hypersetup{
    colorlinks=true,
    urlcolor=blue,
    }
\usepackage{mathtools}
\usepackage{amssymb}
\usepackage{amsmath}
\usepackage{booktabs}
\usepackage{multirow}
\usepackage{caption}
\usepackage{mathtools, nccmath}

\usepackage[T1]{fontenc}
\usepackage{tocloft}
\DeclarePairedDelimiter{\nint}\lfloor\rceil
\captionsetup[table]{name=Tabla}
\DeclarePairedDelimiter\abs{\lvert}{\rvert}%
\DeclarePairedDelimiter\norm{\lVert}{\rVert}%

\usepackage{color}
\usepackage{colortbl}
\usepackage{tcolorbox}
\usepackage{xcolor}
\usepackage{listings}


%%%%%%%%%%%%%%%%%%%%%%%%
    %CODE CONF
%%%%%%%%%%%%%%%%%%%%%%%%
\definecolor{Naranjastring}{HTML}{FF5733}
\definecolor{k3}{HTML}{27A48A}
\definecolor{codeorange}{HTML}{EE7623}
\definecolor{codegray}{rgb}{0.5,0.5,0.5}
\definecolor{codepurple}{rgb}{0.58,0,0.82}
\definecolor{backcolour}{HTML}{003B4D}
\definecolor{funccolor}{HTML}{C1531B}
\definecolor{color-lang}{HTML}{C1C5C8}
\definecolor{colorvar}{HTML}{7398AD}
\definecolor{green--}{HTML}{A0CA92}
%---------------------- C ----------------------- %

\lstdefinestyle{myCstyle}{
  backgroundcolor=\color{backcolour},   
  commentstyle=\color{codeorange},
  keywordstyle=\color{codeorange},
  numberstyle=\tiny\color{codegray},
  stringstyle=\color{green--},
  basicstyle=\ttfamily\footnotesize\linespread{0cm}\color{color-lang},
  breakatwhitespace=false,                         
  captionpos=b,                    
  keepspaces=false,                                                                   
  showstringspaces=false,
  showtabs=false,                  
  tabsize=2,
  inputpath = codigo,
  postbreak=\mbox{\textcolor{red}{$\hookrightarrow$}\space},
  breaklines = false,
  keywordstyle=[2]\color{codepurple},
  keywordstyle=[3]\color{funccolor},
  keywordstyle=[4]\color{colorvar},
  keywordstyle=[5]\color{Naranjastring},
  keywords=[2]{f},
  keywords=[5]{STLIB, LinearSensor},
  keywords=[4]{value, slope, offset, pressure_1, Board, PF11},
  keywords=[3]{main, start, read, update,printf},
  morekeywords = {_delay_ms, sei, ISR, uint8_t, LinearSensor, uint16_t, STLIB, }
  %firstnumber=(auto|last|<number>) para comenzar el conteo de lineas de listing con otro numero. Auto empieza por 1, last, continua la cuenta del listing anterior, number es un numero cualquiera.
}


% set listings
\lstset{%
    basicstyle=\footnotesize\ttfamily,
%   captionpos=t,
    framesep=5em,
    language = C++, style=myCstyle
}

% define backgroundcolor
\definecolor{bggray}{HTML}{003B4D}

% add frame environment
\usepackage[%
    framemethod=tikz,
    skipbelow=\topskip,
    skipabove=\topskip
]{mdframed}
\mdfsetup{%
    leftmargin=0pt,
    rightmargin=0pt,
    backgroundcolor=bggray,
    middlelinecolor=black,
    roundcorner=7
}

\usepackage{etoolbox}% >= v2.1 2011-01-03
\BeforeBeginEnvironment{lstlisting}{\begin{mdframed}\vspace{-0.4em}}
\AfterEndEnvironment{lstlisting}{\end{mdframed}}

% needed for \lstcapt
\def\ifempty#1{\def\temparg{#1}\ifx\temparg\empty}

% make new caption command for listings
\usepackage{caption}
\newcommand{\lstcapt}[2][]{%
    \ifempty{#1}%
        \captionof{lstlisting}{#2}%
    \else%
        \captionof{lstlisting}[#1]{#2}%
    \fi%
    \vspace{0.75\baselineskip}%
}

%%%%%%%%%%%%%%%%%%%%%%%%
        %CODE CONF%
%%%%%%%%%%%%%%%%%%%%%%%%

\newcommand{\Universidad}{Universitat Politècnica de València}
\newcommand{\LogoUniversidad}{LogoUPV}
\newcommand{\Facultad}{Escola Tècnica Superior d'Enginyeria Informàtica}
\newcommand{\LogoFacultad}{Logo_etsinf}
\newcommand{\Titulacion}{Grado en Ingeniería Informática}
\newcommand{\titulo}{\scshape ST-LIB: librería de abstracción de microprocesadores STM32}
% Tipo de trabajo
\newcommand{\tipotrabajo}{Trabajo de fin de grado}
% Datos del autor
\newcommand{\miNombre}{Ricardo Chust Fides}
\newcommand{\miTutor}{ Vicente Atienza}
\newcommand{\colaboradores}{ Daniel González, Stefan Costea, Alejandro Gonzalvo, Pablo González}

% Ubicación
\newcommand{\miUbicacion}{València}
\renewcommand{\thetable}{\arabic{table}}
\renewcommand\spanishtablename{Tabla}
\renewcommand\spanishfigurename{Diagrama}

\begin{document}
\setcounter{secnumdepth}{3}

%%%%%%%%%%%%%%%%%%%%%%%%%%%%%%%%%%%%%%%%%%%%%%%%%%%%%%%%%%%%%%%%%%%%%%%%
% Plantilla TFG/TFM
% Realizado por: Jose Manuel Requena Plens
% Contacto: info@jmrplens.com / Telegram:@jmrplens
%%%%%%%%%%%%%%%%%%%%%%%%%%%%%%%%%%%%%%%%%%%%%%%%%%%%%%%%%%%%%%%%%%%%%%%%

% esta plantilla ha sido ligeramente modificada por Ricardo Chust Fides para
%habilitar CoAutores y Tutor en ella

% Fuentes
\newfontfamily\ArialMT{ArialMT}[Path=./fuentes/] 
\newfontfamily\ArialBoldMT{Arial-BoldMT}[Path=./fuentes/] 
\newfontfamily\ArialBoldItalicMT{Arial-BoldItalicMT}[Path=./fuentes/] 

\begin{titlepage}

% Márgenes de esta pagina modificados
\newgeometry{ignoreall,top=2.5cm,bottom=1cm,outer=2.5cm,inner=2.5cm}

% Pagina sin estilo
\thispagestyle{empty}

%%%%%%%%%%%%%%%%%%%%%%%%%%%%%%%%%%%%%%%%%%
% Universidad, facultad y titulación
\noindent\begin{minipage}{\textwidth}
\centering
% Universidad
 {\ArialMT\fontsize{20pt}{25pt}
 {\addfontfeature{LetterSpace=1.0}
 {\MakeUppercase\Universidad}}}\\[0.5cm]
% Facultad
 {\ArialBoldMT\fontsize{14pt}{17pt}
 {\addfontfeature{LetterSpace=10.0}{
 \MakeUppercase\Facultad}}}\\[0.42cm] 
% Titulación
 {\ArialBoldMT\fontsize{13pt}{16pt}
 {\addfontfeature{LetterSpace=16.0}{
 \Titulacion}}}
\noindent\rule[-0.22cm]{\textwidth}{5pt} % Linea
\end{minipage}

% Relleno hasta logotipos
\vfill

%%%%%%%%%%%%%%%%%%%%%%%%%%%%%%%%%%%%%%%%%%
% Logotipos
% Universidad
\begin{textblock*}{5cm}(0.80cm,-6.52cm)% Ancho - Pos X,PosY
	 \includegraphics[width=3.5cm]{\LogoUniversidad}
\end{textblock*}
% Facultad
\begin{textblock*}{12cm}(5.0cm,-5.82cm)% Ancho - Pos X,PosY
	 \includegraphics[width=11cm]{\LogoFacultad}
\end{textblock*}

% Relleno hasta el título
\vfill

%%%%%%%%%%%%%%%%%%%%%%%%%%%%%%%%%%%%%%%%%%
% Título
\begin{textblock*}{\textwidth}(0cm,-6.1cm)% Ancho - Pos X,PosY
\noindent
\begin{minipage}{\textwidth}
\begin{spacing}{2.1}
 \centering
  {\fontsize{25pt}{30pt} \fontfamily{cmr}\selectfont
  {\addfontfeature{}
    \titulo}}
\end{spacing}
\end{minipage}
\end{textblock*}

% Relleno hasta autor y tutores
\vfill

%%%%%%%%%%%%%%%%%%%%%%%%%%%%%%%%%%%%%%%%%%
% Tipo, Autor y Tutores
\begin{textblock*}{7.35cm}(8.95cm,-6.88cm)% Ancho - Pos X,PosY
\begin{flushleft}
% Tipo trabajo
 {\ArialBoldItalicMT\fontsize{10pt}{12pt}
 {\addfontfeature{}
 {\MakeUppercase\tipotrabajo}}}\\[0.41cm]
% Autor
 {\ArialMT\fontsize{10pt}{12pt}
 {\addfontfeature{}
 Autor:}}\\ 
 {\ArialBoldMT\fontsize{10pt}{12pt}
 {\addfontfeature{}
 {\miNombre}}
 }\\[0.26cm]

 {\ArialMT\fontsize{10pt}{12pt}
 {\addfontfeature{}
 Tutor:}}\\ 
 {\ArialBoldMT\fontsize{10pt}{12pt}
 {\addfontfeature{}
 {\miTutor}}
 }\\[0.26cm]

 {\ArialMT\fontsize{10pt}{12pt}
 {\addfontfeature{}
 Colaboradores:}}\\ 
 {\ArialBoldMT\fontsize{10pt}{12pt}
 {\addfontfeature{}
 {\colaboradores}}
 }\\[0.26cm]

% Localidad y Fecha
 {\ArialBoldItalicMT\fontsize{10pt}{12pt}
 {\addfontfeature{}
 {\MakeUppercase \miUbicacion, \number\year}}}
\end{flushleft}
\end{textblock*}



\end{titlepage} % Fin de portada

% A partir de aquí aplica los márgenes establecidos en configuracioninicial.tex
\restoregeometry




\pagestyle{fancy}
\fancyhf{}
\rhead{\miNombre}
\chead{\begin{picture}(-143,0) \put(-143,0){\includegraphics[width=5cm]{\LogoFacultad}} \end{picture}}
\lhead{\tipotrabajo}
\cfoot{Página \thepage}
\large
\renewcommand*\thesection{\arabic{section}}
\setcounter{section}{0}

\begin{abstract}
\vspace{0.5cm}
\par
Este TFG tratara de documentar tanto el desarrollo como la funcionalidad de la librería ST-LIB para microprocesadores STM32, una librería desarrollada para la competición Hyperloop Week por parte de Hyperloop UPV. El propósito de la librería es abstraer lo máximo posible la implementación del código para las placas que usen microprocesadores STM32, y así reducir notablemente el tiempo y coste de desarrollo de infraestructuras que usen estas placas. \par 
\vspace{0.5cm}
Por su parte el propósito de este TFG sera servir como herramienta a generaciones futuras de Hyperloop y a potenciales usuarios externos para poder entender en profundidad los propósitos de la librería y sus capacidades, así como aprender a usarla rápidamente y tener un referente para futuros proyectos similares. \par 
\vspace{0.5cm}
Dentro del documento se divide en dos grandes partes, una orientada al desarrollo del software de la librería y otra parte a las capacidades funcionales que ofrece como herramienta para ingeniería de computadores. La primera parte tratara las metodologías de desarrollo usadas, la gestión del reparto de trabajo, las decisiones tomadas mientras se construía y sus posibles casos de uso fuera del equipo Hyperloop UPV. La segunda parte documentara la librería para facilitar su aprendizaje, analizara sus capacidades funcionales, tratara posibles expansiones futuras a esta y ofrecerá un análisis empírico de sus costes temporales y espaciales. \par
\vspace{3 cm}

\centering
{\bfseries Abstract \par}
\vspace{0.7 cm}
\raggedright
Resumen pero traducido al ingles //TODO
\vfill

Palabras clave: ST-LIB, STM32, Hyperloop, ingeniería de computadores, desarrollo de software
\end{abstract}


\newpage
\setlength{\cftbeforetoctitleskip}{-3em}
\tableofcontents
\newpage

\section{Objetivo del proyecto}
El propósito principal de la librería es ayudar al desarrollo del vehículo de Hyperloop UPV para la competición de EHW (European Hyperloop Week). Para entender apropiadamente las razones por la que la desarrollamos y las decisiones que tomamos, primero habrá que explicar de forma resumida que es la EHW. \cite{web:EHW:homepage} \par \vspace{0.3 cm}

\subsection{EHW}
La EHW es una competición internacional organizada por múltiples universidades europeas para promover el desarrollo de nuevas tecnologías así como animar a los estudiantes universitarios a meterse en el mundo de la investigación y a ponerse retos cada vez mayores. \par
La competición se divide en dos partes. La primera es la fase de desarrollo, que cubre prácticamente todo el año escolar. Esta parte se basa en la entrega de múltiples documentos técnicos que describen el vehículo así como las nuevas tecnologías desarrolladas para este. La otra parte es la presentación, que sucede a mediados de Julio, durante una semana que da nombre a la competición. Aquí es cuando el trabajo de todo un año da su fruto, y se muestra el vehículo en funcionamiento y se le somete a varias pruebas para mostrar estas nuevas tecnologías que fueron documentadas en la primera parte. Al terminar la EHW queda recoger y comenzar un nuevo ciclo. \par \vspace{0.3 cm}
Lo especial de esta competición es que la fase de desarrollo dura solamente un año. Además, aunque no sea obligatorio presentar un nuevo vehículo cada año, como el objetivo principal de la competición es la mejora y la innovación, es mas que recomendable. Al tener solo un año para preparar un vehículo funcional y que cumpla los requisitos de seguridad de la EHW, el tiempo es un recurso extremadamente importante. \par \vspace{0.3 cm}
Debido a esto, hemos pensado en crear un código reusable, simple de usar, testeado y flexible. Este código es la librería ST-LIB. Su objetivo es, por tanto, ahorrar la mayor cantidad de tiempo al desarrollo de código para la electrónica del vehículo, al mismo tiempo que nos ofrece mayor seguridad al tener un código probado y perfilado decenas de veces. \par
La {simplicidad} es necesaria para que los nuevos miembros no tengan que sufrir una curva de aprendizaje demasiado empinada, que perjudique su instrucción en la materia o incluso los lleve a abandonar el proyecto al ser abrumados por su dificultad. La {reusabilidad} servirá para que no tengamos que desarrollar una librería nueva cada año, y podamos dedicarnos en su lugar a expandirla y mejorarla. Por último necesitaremos que este {testeado} y sea {flexible} para evitar encontrarnos con fallos o necesidades no cubiertas en un futuro al trabajar con la librería. 

\subsection{Uso externo}
Inicialmente la ST-LIB estaba diseñada entorno a las necesidades del proyecto Hyperloop H8, y no se tenía en consideración prepararla para uso público. No obstante, a lo largo de la fase de diseño y desarrollo se fue viendo esta posibilidad cada vez mas factible, debido principalmente a las necesidades de diseño impuestas y a la versatilidad de la HAL; una macro libreria de abstracción de la programación ensamblador para arquitecturas stm32 la cual funciona para todas las familias. \par

Así pues se decidió hacer pública la ST-LIB a final de la competición. El público objetivo es principalmente equipos interesados en la metodología de desarrollo scrum con objetivo de sacar un producto funcional rapidamente. Grandes empresas con objetivos a largo plazo se encontraran con que la ST-LIB no corre código suficientemente rápido en comparación a hard-codear \footnote{Hacer código con el menor nivel de abstracción posible, poniendo la optimización por encima de cualquier otra propiedad del código}, lo cual requerirá procesadores más veloces, que desemboca en mayor coste de producción. El usuario potencial de esta librería son equipos pequeños o bien proyectos sin una cadena de producción; como lo son mocks funcionales, productos de diseño (creados para un único cliente), y herramientas de testeo para la propia empresa. 
\newpage

\section{instalación y uso}
En este apartado se explicara como instalar los programas necesarios para usar la librería, como utilizarla, que se puede configurar e incluso como modificarla por si fuese necesario adaptarla para otros microprocesadores de stm32 o se quisiera en algún futuro ampliarla. Este apartado no explica nada de la librería que no se puedan encontrar en otros apartados y su único propósito es facilitar la instalación y uso de esta. Si no se tiene interés en el uso directo de la librería, este apartado es opcional. 

\subsection{Git}
Para poder instalar la librería hay dos opciones. La primera es ir al github de la ST-LIB y descargar una versión de entre las disponibles directamente, lo cual le dara un comprimido portable que puede descomprimir directamente en su workspace. La segunda es usar git para copiar el repositorio. Mientras que ambas opciones son válidas, es recomendable usar git para facilitar el manejo de versiones, ya que hara mas comodo cambiar entre estas e incluso crear un repositorio aparte para guardar la librería modificada con su propio historial de versiones.
\par
La instalación de Git es muy sencilla\cite{web:git:installationGit} y esta claramente explicada en su página\cite{web:git:installationGitHelp}. La única complicación que se puede dar es que en el caso de Windows no se añada al path correctamente, dependiendo del método de instalación elegido. Si se da el caso, en el buscador de windows escriba``Path'', abra ``Editar las variables de entorno del sistema'', vaya a la pestaña ``Opciones avanzadas'', seleccione ``variables de entorno'', elija ``PATH'', luego presione ``Editar'', se abrira una nueva ventana con varios botones, incluido el boton ``Nuevo''. Se aprieta este boton, y se escribe la dirección en la que se instaló Git. Si no se modificó el lugar de instalación, debería encontrarse en ``C{:}{\textbackslash}Program Files{\textbackslash}Git''. 
\par \vspace{0.3 cm}
Una vez se tiene instalado Git en el sistema operativo se puede usar la consola del susodicho para copiar el repositorio. Simplemente escriba en la consola cd ``path'' (donde path es la dirección donde se quiere copiar la librería, recomendable que se escoja el workspace de su editor de código), presione Enter, asegurese de que haya cambiado la dirección en la que se encuentra a ``path'', y use git clone \textit{https://github.com/HyperloopUPV-H8/ST-LIB.git} para copiar desde el repositorio remoto todo su contenido. Se recomienda también clonar el proyecto base, el cual ya tiene todas las configuraciones implementadas para correr la librería. Para esta use git clone \textit{https://github.com/HyperloopUPV-H8/template-project.git}. \par \vspace{0.3 cm}
Una vez se tenga instalado el repositorio para cambiar de versiones simplemente se usa el comando git checkout \textit{versión} para cambiar la versión de la librería. Una vez todo funcione, es libre de de seguir usando git o trabajar en archivos locales. Se recomienda que copie la instalación fuera de las carpetas git y trabaje en la copia para evitar eliminar todo su progreso en el código al cambiar de versión (git tiene una protección contra esto pero no te permitira cambiar la versión mientras esta protección este levantada)\par
\subsection{}


\par
La librería funciona tanto en Linux, como en Windows, y como en MacOS. En esta explicación nos centraremos en la instalación en Linux (mas recomendable) pero dejare enlaces y referencias a pasos a seguir para los otros sistemas operativos. 
\par \vspace{0.3 cm}
Lo primero que se debe hacer es instalar el programa de stm para la programación de microprocesadores. Este programa se llama stm32cubeIDE, y esta basado en eclipse, un programa de edición de código alternativo a Visual Studio especializado en programación de Java y C/C++. Al estar basado en eclipse, siempre que se quiera hacer algo en stm32cubeIDE que no sea estrictamente de los microcontroladores (como por ejemplo gestionar vinculación de librerías) se puede buscar documentación de eclipse directamente la cual es mucho mas abundante que la del cubeIDE. \par
Para instalar el cubeIDE debemos ir a la pagina oficial de stm32 \cite{web:STM32:installationCube} y descargar un instalador para nuestro sistema operativo. Para Linux hay 3, uno para ubuntu \textit{(deb)}, uno para fedora \textit{(rpm)} y uno genérico. La versión en la que se trabajo la librería fue la 1.10.1, así que esta es la versión preferente. \par
Para ejecutar el instalador en nuestro sistema Linux debemos descomprimir el .zip con unzip, y luego deberemos ejecutar el .sh en la consola dentro de la carpeta donde se haya descomprimido, con el comando \textbf{sudo sh ./st-stm32cubeide\_\textit{tu versión}.sh}. Deberás aceptar las licencias de usuario y indicarle el directorio de instalación. Para Windows y MacOS en lugar de un comando por consola se puede simplemente ejecutar haciendo doble clic al archivo extraído. 
\par \vspace{0.3 cm}
Una vez instalado el cubeIDE deberás añadir la librería para poder usarla. Simplemente selecciona la opción de \textit{``Open projects from file system''}, introduce el directorio donde descargaste (o copiaste desde git) y dale a continuar. \par

tendrás la opción de crear un proyecto nuevo desde cero, que te dejara seleccionar tu microprocesador y te construirá un .ioc con la configuración estándar para este, el cual tendrás que modificar a tu gusto. Aunque crear el proyecto desde cero y luego importar la librería sea posible llevara bastante tiempo debido a que habrá que configurar los periféricos en el .ioc y crear un archivo runes de configuración de la librería, así que para el caso de los microprocesadores stm32h7z23 existe ya un proyecto configurado que se puede importar usando git y la opción de abrir un proyecto desde un sistema de archivos. TODO

\newpage

\section{Desarrollo de la librería}

En esta sección se vera una memoria de como se desarrollo la librería, explicando las condiciones sobre las que se trabajaba, los objetivos principales de la librería \par


\subsection{Métodos de gestión de proyecto}
Para estructurar un proyecto en grupo se necesita una organización clara. La jerarquía del equipo usada fue una estructura organizativa funcional de poca verticalidad. Se requiere de una estructura organizativa funcional debido a que el proyecto Hyperloop es multitudinario, y los sub-sistemas deben saber a quien dirigirse a partir de su competencia cuando requieren a personas de campos específicos. El proyecto de la ST-LIB entraba dentro de el sub-sistema de firmware, y requería de comunicación directa y continua con el equipo de software y hardware para conocer las especificaciones funcionales necesarias que debía cubrir la librería. \par\vspace{0.3 cm}
Esta estructura se dividió en tres capas, los miembros y colaboradores del equipo, los PM \textit{(project manager)} que gestionaban un sub-sistema entero, y los capitanes que dirigían a todo el equipo a partir de comunicarse principalmente con los PM y para casos específicos directamente con los miembros y colaboradores mas instruidos en la tarea. Aunque hubiese una jerarquía de equipo se dejaba abierta la comunicación directa entre todo miembro del equipo, y se usaba la responsabilidad y confianza en los miembros como único muro para no abrumar a los PM y capitanes con dudas y consultas varias. Este tipo de método de trabajo, si se consigue llevar a cabo, es muy ágil y flexible, pero es mas vulnerable a fallos individuales y puede llegar a sobrecargar a miembros del equipo clave si no se trata con cautela. Se requiere de personas capaces de trabajar en equipo, así que la instrucción para este tipo de metodologías es obligatoria para cada nuevo miembro. 
\par \vspace{0.3 cm}
Para el sub-sistema de firmware se uso específicamente una metodología scrum ágil basada en fases de proyecto con sprints e hiatos de planificación para las consecuentes fases y para tratar posibles retrasos. En cada sprint se le asignaba a cada miembro unos trabajos que se debían llevar a cabo en un marco de tiempo, y al final de este se revisaba cuantos de estos objetivos se habían cumplido y se trataban los posibles retrasos así como se aprovechaban los posibles adelantos redistribuyendo trabajo. \par \vspace{0.3 cm}
Esta metodología pone responsabilidad en cada miembro al tener una forma objetiva de marcar su trabajo. Lo bueno de esto es que permite aligerar la carga de trabajo de los PM y capitanes que debido a la estructura de equipo corren riesgo de sobrecarga. Lo malo es que al ofrecer una parte del trabajo a cada miembro y planificar teniendo en cuenta lo que van a completar un miembro disfuncional, ausente, o que sufra un contratiempo puede provocar un fuerte cuello de botella y dañar el ritmo del equipo entero. \par
Para mitigar lo máximo posible la rotura del ritmo de trabajo, los sprints se reducen a una semana de tamaño, haciendo reuniones a mediados y final de sprint para redistribuir el trabajo y analizar los avances, permitiendo rectificar problemas en el avance del trabajo en pocos días, siempre y cuando los miembros sean comunicativos.
\par \vspace{0.3 cm}
En cuanto a las fases del proyecto, que definen hitos importantes del avance del proyecto, se dividen en fase de diseño, fase de testing y fase de presentación. La fase de diseño incluye la construcción del equipo, la comunicación de requisitos, el diseño de la estructura de la librería y la implementación mínima funcional de esta, en ese orden. Como esta fase de diseño sucede a la vez que la fase de diseño de otros sub-sistemas, es común la retroalimentación de requisitos y la reestructuración de la librería mientras se esta en fases mas avanzadas, por problemas que se puedan encontrar o mejoras posibles que se puedan añadir. Ahí es donde brilla la naturaleza iterativa del scrum, permitiendo añadir a los sprints cambios de la estructura y comunicarlos rápidamente al equipo sin afectar al ritmo de trabajo. \par \vspace{0.3 cm}
Una vez se termina la fase de diseño se hace un release de la librería (mas adelante se elaborara en el concepto de release) con una funcionalidad probada mínima y puede comenzar la fase de testing. Esta fase es la mas intensa debido a que es aquí donde se encuentran la mayoría de errores y contratiempos. Se basa en utilizar la librería para implementar múltiples placas con requisitos funcionales críticos usando todas las herramientas de la librería y hacer pasar a estas placas unos tests tanto de seguridad como de capacidades funcionales. \par
Es posible que en esta fase se requieran cambios en el diseño de la estructura de la librería como el añadido de nuevos módulos o mejorar las capacidades de módulos ya existentes, lo cual puede afectar a su vez a otros módulos de la librería provocando un efecto cascada de trabajos urgentes y fallos en la librería. Por ello en esta fase se requiere de horarios flexibles, una comunicación muy abierta, y de trabajo preventivo. En esta fase se hacen múltiples releases que arreglan errores, añaden funcionalidades y confirman revisiones al código. Al terminar esta fase la librería debe estar en su versión final, debe cumplir todos los requisitos y estar lista para actuar en sistemas críticos. \par \vspace{0.3 cm}
La ultima fase es la fase de presentación, que dura alrededor de un mes. Aquí se junta el trabajo de todos los sub-sistemas, se termina el producto, se planifica las formas de actuar a la hora de la presentación y finalmente se presenta en la EHW. Esta fase seria la equivalente a los meses después de la salida al mercado de un producto, donde el equipo recibe feedback y hace unos cambios mínimos de ultima hora para arreglar fallos que no se pudieron ver sin tener el producto final. Si por alguna razón se requiriesen cambios drásticos en la librería, querría decir que la metodología de trabajo ha fallado. 

\subsection{Gitflow}
El Gitflow es una rama de la metodología de gestión de proyectos orientada principalmente a el control de archivos y documentos del proyecto, pero como esta parte es tan esencial en un proyecto de desarrollo de software como lo es la creación de una librería, se le ha dado un apartado propio para evitar saturar la sección de métodos de gestión de proyectos. 
\par \vspace{0.3 cm}
Gitflow es una estructura y planificación de el uso de programas de gestión de versiones (principalmente Git, del cual viene su nombre) que define los pasos a seguir para añadir cambios o introducir nuevos archivos al espacio de trabajo del equipo sin interferir en el trabajo de otros.
El Gitflow se basa en el uso de ramas \textit{(o ``branches'')} de trabajo que se encargan de controlar partes especificas de los documentos (en nuestro caso, los módulos de la librería) que hacen una instantánea de el estado actual del espacio de trabajo y te deja modificar esta instantánea sin afectar al espacio en común del equipo, y por ende evitando daño directo al avance de los compañeros. Una vez se ha terminado de trabajar en la rama, se debe reunir los avances hechos con el espacio de trabajo del equipo, en un acto conocido como fusión de las ramas \textit{(o ``merge'')}\par \vspace{0.3 cm}

En nuestro espacio de trabajo en común esta compuesto por dos ramas, \textbf{development} que guarda todos los avances que fueron mergeados, y \textbf{main} que guarda la ultima instantánea del proyecto que ha sido correctamente testeada y su funcionalidad esta asegurada. \par
Para poder juntar los avances hechos por un miembro del equipo con development \textit{(la rama de trabajo en común)} se debe tener un mínimo de control para asegurar que no se ha cometido un fallo que comprometa la funcionalidad de la rama development. Para ello se usa las \textbf{pull-request}, una funcionalidad de Git que permite a otros miembros del equipo que no hayan trabajado en esa rama revisar los cambios, pedir modificaciones a la rama a fusionar, y finalmente aceptar los cambios. En nuestro caso en concreto, se requería que la mitad del equipo de firmware (2 personas + el propietario de la rama) aceptara la pull-request de la rama. \par
Para hacer una nueva release \textit{(fusionar development con main)} se reúne al equipo entero en una reunión especial y se hace una pull-request que requiere de la aceptación del 70\% del equipo (todos menos 1) para completar esta. 
\par \vspace{0.3 cm}

Una vez la librería esta completa comienza la fase de testeo, donde el Gitflow se modifica para facilitar lo máximo posible la salida y aprovechar las nuevas herramientas que se fueron desarrollando en paralelo a la librería. Para ello lo primero que se hace es reducir el numero de personas que aceptan la pull-request para mergear a main en 1 mas la herramienta de testeo automático (un servidor desarrollado para la fase de testing que corre pruebas en el código del pull request para comprobar que el código no se rompió en los cambios de esta, funcionando 24 horas). Esto reduce mucho la dependencia entre los programadores de la librería permitiéndoles sacar mucho mas rápido los fixes. \par
Además se implemento un control de versiones de la rama main, permitiendo tener varias versiones funcionando al mismo tiempo; principalmente para el caso en el que una placa funcionase completamente con una versión pudiera congelarse sin tener que revisarse para las nuevas versiones, aunque también como seguro en caso de que un merge estropeara la librería de una forma imprevista ya que el control sobre la rama main fue reducido a favor de mayor facilidad a la hora de revisarla. \par
El último cambio al Gitflow entre las fases de desarrollo y testeo fue añadir como requisito a las actualizaciones a main un log file que explicara los cambios que se han producido en la librería a nivel funcional con cada nueva actualización. Estos cambios además no debían afectar a la interfaz de la librería a menos que fuese absolutamente necesario. 
\par \vspace{0.3 cm}

En cuanto a los programas usados para aplicar el Gitflow los principales fueron Git, GitHub, Visual Studio Code, GitKraken, y el formato MarkDown para la escritura de la Wiki del proyecto. En Git se usaban dos repositorios separados, uno para la propia librería y otro para el proyecto de las placas que servía, entre otras cosas, para probar la librería. Había una dependencia unidireccional entre la librería y el proyecto de los microcontroladores, por lo que el segundo debía actualizarse rápidamente a los cambios del primero. Para modificar este ultimo, se usaba un Gitflow mas simplificado sin pull-requests, permitiendo cambiar mucho mas rápido el proyecto de las placas a cambio de tener el riesgo de que este fallara. Este riesgo era únicamente aceptable gracias a que la librería no dependía del susodicho, y por tanto los fallos en el proyecto de las placas no afectarían gravemente al avance del trabajo. \par
Respecto a GitHub, Visual Studio Code y GitKraken eran soportes de alto nivel para facilitar el uso y entendimiento de Git a los miembros del equipo menos versados en el uso de esta herramienta. Visual Studio Code permite hacer cambios rápidos y controlar conflictos de forma mas visual en las ramas, y tiene atajos para las acciones mas usadas dentro de Git. GitKraken permite observar de una forma mas \textit{``user friendly''} los movimientos entre ramas para ver si estaban habiendo problemas a la hora de aplicar el concepto del Gitflow en la practica. Se podían ver de que ramas a cuales se habían hecho los merge y los pull-request, permitiendo alertar branching excesivo y resaltando las ramas mas problemáticas. GitHub cerraba el circulo ofreciendo control sobre las pull-request, los comentarios, y dando la capacidad de modificar la estructura del Gitflow en minutos en caso de necesidad.
\par \vspace{0.3 cm}
Los módulos de la librería (que en el Gitflow son tratados como ramas) se separaron en tres grandes bloques según su nivel de abstracción: Core, Low y High. Cada nivel requería de distintos conocimientos y cantidad de comunicación, además de que cada nivel superior tenia dependencias con los niveles previos. Para que esto fuese posible, se requiere que el grafo de dependencias entre módulos de la librería cumpla unas normas, debe ser un grafo multinivel, cuyos niveles sean los tres bloques. Esto quiere decir, a grosso modo, que ningún modulo de la librería puede depender de otro modulo del mismo nivel o de un nivel superior. Diseñar la librería así tiene otra gran ventaja añadida, y es que mientras se siga este concepto de diseño es imposible sufrir dependencias cíclicas, pues para que un grafo sea multinivel debe, entre otras cosas, ser acíclico. \par
La ST-LIB Core es el bloque que requería mas conocimientos de hardware y firmware para ser creada, abstraía directamente los registros de Hardware, apoyándose únicamente en la HAL, la cual requiere de conocimientos tan amplios como trabajar directamente con los registros. Esta es la que tendría mas peso en la eficiencia del código, la seguridad, y la flexibilidad de este. La mayor ventaja es que es un punto de apoyo para las capas superiores y no esta diseñada con la intención de ser usada directamente, así que podía ser tan obtusa y fea en su uso como fuese necesario para cumplir los requisitos dados por los otros sub-sistemas. \par
La ST-LIB Low es la capa intermedia entre el usuario y el corazón de la librería, y su objetivo principal es traducir a conceptos de lenguajes mas abstractos las herramientas dadas por la ST-LIB Core. Pasar de uso de ids y registros a estructuras, clases y POO, gestionar los posibles fallos y recuperarse de errores o datos aberrantes, ayudar a debuggear el código al usuario, y hacer macros de estimaciones y cálculos usados en múltiples capas son algunas de las funciones que tiene esta capa. \par
Por ultimo, la ST-LIB High tiene como objetivo juntar todos los modulos de la ST-LIB Low en grandes macros como \textit{``iniciar''}, \textit{``terminar''}, \textit{``bucle de trabajo''}, o \textit{``interrupción''}, así como activar o desactivar los modulos que están en uso, y imponer protecciones entre modulos, y facilitar la creación de macros especificas para cada placa. 
\newpage

\section{Diseño y funcionalidad}
En esta sección se explicara en mas detalle la estructura de la librería, como funciona y cuales son los casos de uso esperados para cada uno de sus modulos. Se dividirá en las tres partes principales de la librería: Core, Low y High, mas una explicación generalizada del objetivo de diseño. 
\subsection{ST-LIB Core}
\subsubsection{Modelos y Servicios}
La ST-LIB Core se divide en la parte estructural (o los modelos) y la parte funcional (o los servicios). Los modelos permiten abstraer el concepto de registros de periféricos, relojes, contadores, uso de la memoria flash, y demás propiedades del propio hardware del micro que pueden ser demasiado abrumadoras para una persona que se esta introduciendo al mundo del firmware y demasiado repetitivas y laboriosas para un equipo que ya conoce sus necesidades sobre estos aspectos del hardware y prefieren tener una base ya creada para no tener que configurar cada proyecto que hagan. \par \vspace{0.3 cm}

Los servicios por su parte abstraen las macros mas comunes en el mundo del firmware, como iniciar periférico, crear interrupción, atender interrupción, reconfigurar reloj, crear alarma, atender alarma, y lecturas y escrituras de todo tipo. Los servicios de la ST-LIB Core son primitivos y de un nivel de abstracción muy bajo, y solo existen como soporte para niveles de abstracción superiores y como opción a configurar para expertos de firmware. Las funciones que hacen son muy simples, pero reducen la necesidad de 9 lineas de código a 1, y gestionan los errores mas comunes cuando se hacen proyectos a gran escala con múltiples micros como lo son no haber declarado o iniciado un Pin, equivocarse en un numero y llegar a valores peligrosos para el micro, o usar un Pin configurado de una forma para una función que requiere otra configuración. \par \vspace{0.3 cm}

La mayoría de veces gestiona estos errores a través del ErrorHandler, una clase que se dedica específicamente a dar información retro-alimentada al programador a partir de mensajes en el protocolo de comunicación preferido (Viene configurado para usar el protocolo FD-CAN, pero con un cambio a dos lineas de un método se pueden usar cualquier otro de los protocolos ofrecidos por la librería), y de parar el micro en caso de fallo de una forma mas segura en lugar de usar un \textit{Hard Fault}. 

\subsubsection{Pin}
La clase Pin es el primer nivel de abstracción sobre la librería HAL y el código C con registros. Es una estructura de datos cuyo propósito es que los usuarios puedan identificar el Pin rápidamente en el datasheet y escribirlo directamente en el código sin tener que hacer traducciones ni convocar métodos. Esta pertenece a los Modelos de la ST-LIB Core.
\par \vspace{0.3 cm}
Esta estructura se encarga de abstraer el concepto de los pines de la micro en su totalidad. Lo primero que hace es mapear los valores de registro de los pines a unos valores mas entendibles para el humano, usando los nombres que se le dan en la núcleo para ello. Por ejemplo, si a la configuración del pin A5 se accede con el registro \textit{0x400000001802000000000020} permite al usuario obtener este registro dando los valores A y 5 a los enum\footnote{Un enum es una estructura de C que se dedica a enumerar distintos valores para distintos nombres, similar al concepto de una variable o un mapa pero se soluciona en tiempo de compilación, por lo que no afecta a la velocidad del programa}   GPIOPin y GPIOPort. 
\par \vspace{0.3 cm}
También mapea de la misma forma las posibles configuraciones que puede tener un Pin. Para eso utiliza hasta tres enum mas; uno para saber si esta ya configurado, otro para saber que configuración tiene, y uno tercero para indicar configuraciones especiales en caso de que los modos mas comunes no sean lo que se busca. 
\par \vspace{0.3 cm}
Su ultima funcionalidad es la capacidad para registrar e iniciar cada uno de los pines a partir de los métodos inscribe() y start(). Estos métodos también añaden una capa de abstracción adicional ya que inscribe acepta directamente el nombre del pin y lo traduce a los valores necesarios para obtener su dirección de registro. Con este método se puede usar directamente inscribe(A5,ANALOG) en lugar de tener que obtener la dirección de memoria de la configuración usando A+5 y usar un writemem para introducir la configuración a mano. El método start() usa los valores introducidos en el inscribe para cada Pin para completar la configuración añadiendo todos los extras necesarios, como lo son DMA, relojes y la asignación de espacio de memoria para guardar las variables. Si start() se quedara sin relojes o canales DMA para asignar a los pines configurados usaría el ErrorHandler para avisar al programador, pero esta probabilidad es ínfima pues requeriría configuraciones muy especificas (\textit{como ya explicaremos en la clase Time}). \par \vspace{0.3 cm}
Como añadido adicional, existen construidos ya todos los pines que hay disponibles en los micros de la familia H7 como estructuras de datos publicas, con los alias de P+\textit{puerto}+\textit{pin}. Por ejemplo, el pin A5 es accesible a través de un extern\footnote{extern es una palabra clave de C++ que permite indicarle al compilador que te refieres a una instancia de la variable que ya existe en otro documento, en lugar de querer crear una instancia nueva con el mismo nombre en el documento actual} como \textit{PA5} 

\subsubsection{DMA}
El concepto de Direct Memory Access (DMA) viene de la necesidad en sistemas críticos de aliviar la carga del procesador. Como un porcentaje considerable del tiempo de ejecución es consumido gestionando los accesos a memoria y los datos recibidos a través de los periféricos, se decidió que seria mas efectivo crear una unidad especial de asistencia al procesador en estas acciones antes que gastar recursos en aumentar mas la potencia de este. \par \vspace{0.3 cm}
DMA es entonces la abstracción del uso de un hardware especializado que puede conectarse directamente a la memoria para transferir datos por un bus. En un principio solo hacia un acto muy primitivo, mover datos de un lugar a otro, en un rango de memoria limitado con la opción de ciclar dentro de este rango de memoria o terminar su ejecución y requerir reconfigurarse cuando llenase el buffer. Sin embargo, viendo lo efectivo y barato que era implementar los DMA, con los años se fue añadiendo mas canales de direct memory access a los microprocesadores y se les dio mayor capacidad de calculo, permitiendo incluso aplicar cálculos simples o relegar funciones a otras unidades de hardware y esperar su respuesta para guardarla dentro de lugares de memoria mas específicos. 
\par \vspace{0.3 cm}
Mientras que el uso de DMA no es necesario para hacer nada, aprovechar esta tecnología puede mejorar mucho la eficiencia de los procesos del micro al poder ceder decenas de lineas de código que gestionan información de forma iterativa de la unidad de proceso a las unidades de DMA especializadas. Por ello, se necesitaba una abstracción de los canales DMA en la librería para poder aumentar su eficiencia en todos los periféricos que pudiesen hacer uso de esta. \par \vspace{0.3 cm}
El modelo de la DMA de la librería es bastante simplista, y deja mucha libertad y cosas por configurar a las clases de mayores niveles de abstracción ya que esta utilidad es aplicable a una plétora de casos y no es estrictamente necesaria para ninguno de estos. Se dedica a definir con enum los canales de DMA, usar un mapa para guardarse los canales libres y los canales usados, y ofrecer un método de inscripción de DMA que asigna el primer canal libre o permite, en su lugar, elegir al programador uno especifico. Luego tiene el método start() del que prácticamente todas las clases de la ST-LIB Core gozan y simplifica el inicio de estos servicios. 

\subsubsection{Paquetes}
La estructura de paquetes puede ser probablemente la clase de la librería mas difícil de comprender a primera vista. Debido a la complejidad de abstracción del concepto de paquete conservando su polimorfismo, el modelo packets (paquetes en ingles) usa templates\footnote{los templates son una palabra clave de C++ para funciones, variables y clases por igual que permite decirle al compilador que el objeto al que se le asigna funciona para mas de un tipo de variable. Los templates pueden indicar que funciona para todas las variables numéricas, para todas las variables que guarden un valor, o para variables dentro de una clase definida por el usuario, permitiendo una gran expresividad y polimorfismo sin requerir centenas de lineas de código} infinitos recursivos para auto estructurarse en compilación dependiendo de las necesidades vistas en el código. 
\par \vspace{0.3 cm}
Como esta clase es tan compleja, esta diseñada inicialmente como una \textit{``caja negra''}, es decir, no hace falta entender como funciona para poder usarla. Simplemente al declararse un paquete debe indicársele que tipos de variable guarda y cuantas, y luego usarse un método para o bien recibir un paquete desde un periférico y guardarlo dentro de la variable paquete creada; o bien meter valores dentro de la variable paquete (que también se puede hacer durante la construcción mediante su constructor), y enviar el paquete a través de uno de los periféricos con los servicios de comunicación que mas adelante se verán. 
\par \vspace{0.3 cm}
Sin embargo, aunque no haga falta comprender el modelo Packets para poder usarlo, en el resto de esta sección se tratara de explicar como funciona. Si no se tiene comprender su funcionamiento o si se tiene una idea suficiente de su estructura con la explicación de arriba, puede pasar a la siguiente sección. En el caso de que quiera entender como funciona la explicación comienza en el siguiente parágrafo. 
\par \vspace{0.3 cm}
Como se ha mencionado arriba, el modelo de paquetes debe ser capaz de almacenar cualquier tipo o tipos de variable, en cualquier tipo de combinación, y guardarlos como un objeto paquete. A nivel de programación, esto quiere decir que debe ser capaz de recibir como parámetro cualquier combinación de tipos de variable, traducirlos a un valor binario para poder enviarlas a través de cualquier protocolo de información, y luego poder traducir de vuelta ese valor binario a los tipos de variable indicados para cerrar el concepto de comunicación por paquetes. Se requiere entonces que sea un modelo template para poder recibir cualquier tipo de variable que se pueda traducir a binario \textit{(como a nivel de hardware todo esta traducido a binario, se sabe que cualquier variable tiene al menos una forma de traducirla a binario)}, y se requiere que pueda hacer pattern matching infinito pues de antemano se desconoce cuantos parámetros van a darle al paquete. \par \vspace{0.3 cm}
Todos los métodos están sobrecargados para dos tipos de parámetros, o bien cualquier cantidad de parámetros mayor o igual que 1, o bien ningún parámetro. Este diseño esta hecho a propósito para poder definir las funciones de los métodos como series geométricas, con su caso base (0 parámetros) y luego el caso de n parámetros. Al igual que las series geométricas, una función se define por una cantidad de operaciones más la misma función pero de n-1 valor (o parámetros, en este caso). Así, un paquete de 4 parámetros se define por la traducción del primer parámetro mas la definición de un paquete que contenga los otros tres parámetros. Se traduce el primer parámetro, y se convoca un paquete con tres parámetros que traduce el primero de sus parámetros y convoca a un paquete con dos parámetros. Así hasta llegar a 0 parámetros, donde se retorna el valor base (que es nada, pues el paquete esta vacío) y cuando termina su convocación el paquete de un parámetro concatena su traducción a la del paquete de 0 parámetros, y termina también la convocación del paquete de un parámetro, llegando a hacer una cadena recursiva desde los parámetros requeridos hasta 0 y de vuelta a los parámetros requeridos. Básicamente estos templates permiten hacer declaraciones recursivas en tiempo de compilación (luego en tiempo de ejecución también deberán hacerse convocaciones recursivas debido a que la palabra clave inline no funciona con los templates, pero solo sera necesario en la construcción del paquete y en las traducciones)
\par \vspace{0.3 cm}
Ahora pasaremos a explicar como maneja los castéos de las variables a binario. Aunque todas las variables tengan al menos una forma de transformarse a binario, esto no quiere decir que esta forma este codificada para el uso del programa o que sea la forma de castéo mas sencilla y eficiente. Por ello, la clase PacketValue sirve de soporte para la clase Packet y maneja los posibles valores que puede recibir como parámetros dependiendo de algunas propiedades que puedan requerir. Esta es una estructura de tipos que define grupos de tipos y dependiendo de a que tipo pertenezcan usa un método de traducción u otro. Los separadores de grupos definidos son isContainer y isIntegral, generando los grupos Container, Integral y CustomButNotContainer. Los que pertenecen al tipo isIntegral tienen una traducción directa a binario dada por C que esta demostrada ser la mas efectiva. Los que pertenecen a isContainer quiere decir que en verdad guarda mas de una única variable, como una array, un mapa o un paquete (ofreciendo la opción de nestear paquetes dentro de paquetes). 
\par \vspace{0.3 cm}

\subsubsection{Periféricos básicos: Analog y Digital}
Analog y Digital en concepto son sencillos, y la mayoría del tiempo se pierde en configurarlos. Por ello, en la librería Core hemos creado las clases Analog, DigitalInput, DigitalOutput y EXTI principalmente para facilitar esto. \par
La estructura de los tres primeros es la misma: mapas que alocan los recursos de la placa y los metodos inscribe, start, y get value/set state que permiten configurar un pin (ya definido en la clase Pin) y recibir la id de la configuración, inicializar todos los pines, y hacer una orden de lectura/escritura en los periférico. En este nivel aun se deben gestionar id's pero en la ST-LIB Low se pone una capa mas de abstracción para tenerlo como un objeto de una clase. 
\par \vspace{0.3 cm}
Para utilizarlo es tan sencillo como parece, primero convocas inscribe dandole como parametro el Pin que deseas configurar, y guardando su retorno en una variable uint8, luego convocas el start y por último tomas las lecturas cada vez que las necesites. Aquí un ejemplo del main:

\begin{lstlisting}
uint8_t DigitalID = DigitalInputService::inscribe(PA5);
uint8_t AnalogID = ADC::inscribe(PA0).value(); #devuelve un optional
uint8_t OutputID = DigitalOutputService::inscribe(PC0);
HALAL::start(); #macro que contiene todos los metodos start necesarios
while(1){
  DigitalOutputService::set_pin_state(1);
  printf(ADC::get_value(AnalogID));
  printf(DigitalInputService::get_value(DigitalID));
}
\end{lstlisting}
\par \vspace{0.3 cm}
Se recomienda utilizar el HALAL start en lugar de usar los starts de los modulos ya que algunos modulos dependen de que otros hayan iniciado y tienen que activarse en un orden en concreto. Por ejemplo, el modulo ADC usa las DMA para capturar los valores sin usar capacidad de procesador para mejorar la velocidad del codigo, pero tiene el pequeño defecto de no funcionar si las DMA no 
\par \vspace{0.3 cm}
Aparte de estos tres sensores esta el EXTI (External Interrupt) que permite activar interrupciones directamente a partir de cableado. El EXTI es mas especial porque le puedes especificar que es lo que tiene que hacer cuando sea interrumpido a traves de una \textit{lambda expression}. Las lambda son basicamente funciones sin identificador, que se define por su clausula de captura de parametros, los parametros que recibe, y el codigo a ejecutar. 
\begin{lstlisting}
uint8_t externalID = ExternalInterrupt::inscribe(PA5,()[]{printf("Hola");},FALLING);
HALAL::start();
\end{lstlisting} 
\par \vspace{0.3 cm}
Este código define un external interrupt en el pin PA5 que imprimirá ``Hola'' cada vez que caiga la señal; es decir, que pase de 1 a 0. Puedes ponerle las opciones de RISING o FALLING, que quiere decir de 0 pasa a 1 y de 1 pasa a 0. \par\vspace{0.3 cm}
Hasta ahora nos hemos referido a la función \textit{printf} como la función que imprime un texto, pero como el periférico por el que se imprime depende de la preferencia del programador, probablemente sea necesario sobreescribirla. Actualmente en la ST-LIB lo imprime por UART, un protocolo de comunicación del cual hablaremos en las próximas secciones. Dependiendo del modo de programación (Nucleo o Board) estara en el uart1 o en el uart2 (definidos en el Runes) \par
La librería la sobreescribe en el modulo UART (sobreescribe el metodo write el cual es convocado por printf), así que para sobreescribirla debe removerse de UART primero.
\subsubsection{CORDIC}
El modulo CORDIC es una librería que aprovecha la unidad de calculo interna con el mismo nombre. Es una librería de aceleración de calculo trigonométrico la cual supera incluso a la librería arm math hecha para calculo por estimación en arquitecturas de microprocesadores. La razón por la que esta pequeña unidad la supera es porque esta diseñada específicamente para aplicar el algoritmo de Volder, especializado en funciones trigonométricas e hiperbólicas. \par
La mayoría de las familias de microprocesadores STM incluyen la unidad de aceleración CORDIC, pero si el micro no lo incluye esta librería no funcionara y se deberá desactivar el módulo dentro de la HAL (normalmente si tu microprocesador no tiene CORDIC la flag de CORDIC MODULE esta ya desactivada, apareciendo el código dentro de los ifdef en grís).
\par \vspace{0.3cm}
Una de las peculiaridades del CORDIC es que solo funciona con lógica binaria; es decir, no acepta coma flotante (ni floats ni doubles). El módulo CORDIC ya viene con un metodo que transforma las float (en radianes) a int32 en el formato que el CORDIC recibe. Sin embargo, no se recomienda abusar de este método ya que la transformación de int a float consume muchos recursos desde el punto de vista de la optimización, haciendo que vaya mas lento incluso que la librería math arm con lógica de floats, lo cual le quita el sentido a usar el módulo CORDIC en primer lugar. \par
Por ello CORDIC solo se puede usar de forma eficiente si tienes tres o mas operaciones trigonométricas consecutivas que aplicar a un valor antes de tener que recastearlo o si se usa lógica binaria en todo el código conectado a los cálculos trigonométricos. El primer caso se vería así:
\begin{lstlisting}
double angle = pi * 0.5;
int32_t unary = RotationComputer::radian_f32_to_q31(angle); 
int32_t *pointer1 = &unary;
int32_t *pointer2;
int32_t *pointer3;
RotationComputer::cos_and_sin(pointer1, pointer2, pointer3, 1);
RotationComputer::phase(pointer2,pointer3,pointer1,1);
angle = RotationComputer::q31_to_radian_f32(unary);
\end{lstlisting} 
\par \vspace{0.3cm}
El método cos and sin recibe un ángulo a traves de su primer parametro y calcula el coseno y el seno, metiendo estos valores en los punteros out que recibe como parametros. La razón por la que usa punteros es porque puede funcionar también con arrays de valores. El último valor es, de hecho, el tamaño de la array que recibe. La otra forma de usarlo, y la mas recomendable si tus periféricos te devuelven en int los valores a procesar (como suele ser el caso), es con la lógica de ints directamente:
\begin{lstlisting}
int32_t *values = values_array;
int32_t *x;
int32_t *y;
RotationComputer:cos_and_sin(values,x,y,values_array_length);
RotationComputer::phase(x,y,values,values_array_length);
\end{lstlisting}
\par \vspace{0.3cm}
Lo único tener en cuenta que hay que traducir el formato en el que se reciba los valores a el formato unario para que el cálculo sea correcto. El formato de los angulos recorre todos los valores posibles de las int, donde $180^o$ es el valor máximo que puede tener la int y $-180^o$ el valor mínimo (máximo negativo). \par
Esto tiene las ventajas de que aprovecha al máximo la precisión que le dan y que las sumas de angulos no sufren overflow, ya que la representación de los angulos negativos funciona igual que la de las ints. Por ejemplo, max int +1 es igual a -max int, y $179.99995^o + 0.0001^o = 180.00005^0 = -179.99995^o$. \par
El formato de las coordenadas simplemente representa un espacio acotado en -1 y 1 tanto en x como en y. Cuando superas el límite simplemente sufre overflow. \par \vspace{0.3cm}
Los metodos phase y modulus operan en dicho espacio. \textit{phase} mide el ángulo entre la recta formada por el punto dado y el punto de origen con el vector (1,0). Esto es especialmente útil para comparar dos angulos, ya que la fase entre ambos es \textit{``la fase del primero con el eje x - la fase del segundo con el eje x''}. Osea, se les hace fase a ambos vectores y se restan los resultados. \par
\textit{modulus} sirve para obtener la distancia entre el origen y el punto dado como un valor escalar. Como esta distancia puede ser mayor que uno (el punto (1,1); por ejemplo), y como este numero solo puede ser positivo, devuelve una uint32 en lugar de una int32 para poder llegar hasta el 2, cambiando su imagen de [-1,1] a [0,2]. Esto es importante porque \textbf{el modulus tiene sus propias unidades}, por lo cual no se puede utilizar sin mas como unidad en el espacio. Si guardas el valor de retorno en una int en lugar de una uint y el resultado era mayor que 1 te devolvera un número negativo, lo cual es imposible pues las distancias (escalares) son positivas necesariamente. 
\subsubsection{PWM}
El modulo PWM (\textit{Power Width Modulated}) es un periférico especial que permite generar una señal periódica cuadrada con un periodo y un ciclo de trabajo \footnote{tiempo que se pasa activo dentro del periodo, en porcentaje. También se le conoce como \textit{duty cycle}} configurables. Estas señales son especialmente útiles para controlar motores, sincronizar comunicaciones y simular una señal análoga con cualquier voltaje deseado. 
\par \vspace{0.3cm}
El módulo PWM tiene una clase con constructor, por lo que para usarla debe guardarse el objeto cuando haya sido instancializado. Ademas del constructor, tiene los métodos turn on, turn off, set frequency, get frequency y set duty cycle. Estos metodos activan el PWM, lo desactivan, cambian la frecuencia (y con ello el periodo) con una uint dada, obtener la frecuencia que tiene guardada, y cambiar el duty cycle. \par
Los periféricos de esta clase no se inicializan en el HALAL start como protección, ya que entre otras cosas las PWM sirven como control de motor, e inicializarlo de forma discreta podría ser peligroso. En su lugar, el usuario debe activarlo y desactivarlo cuando vea necesario. 
\par \vspace{0.3cm}
Ademas de la clase PWM, el módulo tiene tres clases que heredan de ella que añaden nuevas propiedades. Estas son PhasedPWM, DualPWM y DualPhasedPWM. PhasedPWM te permite añadir una fase angular a la onda cuadrada que genera el PWM, para sincronizar la onda con otras ondas a conveniencia. La DualPWM crea dos señales PWM negadas entre si. Y la DualPhasedPWM es una DualPWM a la cual se le puede aplicar fase (ambas son afectadas por el cambio ya que una es la negada de la otra). La fase es una float que puede ir desde -100\% hasta +100\%, que a nivel de onda representa un cambio de fase de -pi hasta +pi (en radianes). Un ejemplo rápido de uso:
\begin{lstlisting}
  DualPhasedPWM pwm(PA5,PA6);
  pwm.set_duty_cycle(40.0);
  pwm.set_frequency(1);
  pwm.set_phase(30.0);
  pwm.turn_on();
\end{lstlisting}
\par \vspace{0.3 cm}
Con este código hemos puesto dos pwm negadas entre si en un duty cycle del 40\%, una frecuencia de 1 Hz, y una fase del 30\% (+pi*0.3 de fase angular). Como se puede comprobar operar las PWM se ha vuelto mucho mas sencillo ya que el módulo gestiona automaticamente toda la configuración de software.  

\subsubsection{Time}
La clase time incluye la gestión de los timers, las alarms, y el rtc (\textit{real time clock}). Como hay una cantidad limitada de timers en el microprocesador y algunos de estos son necesarios para hacer funcionar otros modulos como las PWM, en su lugar se utilizan tres timers generales para que gestionen la gran mayoria de las alarmas y una interfaz con los timers restantes en caso de querer una alarma crítica. \par
Los timers generales low y mid estan configurados para crear una interrupción cada 50 microsegundos y cada milisegundo, respectivamente. Luego de saltar la interrupción comprueba si las condiciones de las alarmas configuradas por el usuario se han cumplido, si es el caso ejecuta la lambda expression que le corresponde y en caso contrario la ignora. Un ejemplo de configuración:
\begin{lstlisting}
  uint64_t counter = 0;
  Time::register_mid_precision_alarm(100,[&](counter){counter++;});
  Time::register_low_precision_alarm(100,[&](counter){printf(counter);});
  HALAL::start();
\end{lstlisting}
\par \vspace{0.3 cm}
Este código sumara cada 100 microsegundos +1 al contador y lo imprimirá cada 100 milisegundos. Hay que indicar que el divisor del mid precision redondea hacia abajo, asi que poner menos de 50 microsegundos como intervalo de la alarma hara que no funcione correctamente. \par



\section{Testeo y profiling de la librería}
\subsection{Testing automático}
Durante la fase de desarrollo y parte de la fase de testeo se preparo una herramienta especial para poder hacer pruebas continuas a la librería conocida como la ATP (\textit{Automatic Testing Platform}). La ATP se divide en una pieza de hardware especializada que emula múltiples señales utilizadas en casos de uso reales denominada SHUTP y una OrangePi 5 con ubuntu como RTOS\footnote{RTOS se refiere a Real Time Operating System, o sistema operativo, en el contexto de los microprocesadores} que hace de servidor para recibir los test a probar y enterarse de las pull-request abiertas en GitHub de las cuales obtener el código, a la cual simplemente denominaremos OrangePi. \par \vspace{0.3cm}

La SHUTP se compone de una placa con 400 pin outs\footnote{Los pin outs son unas conexiones de circuitería que permiten conectar un componente específico de una placa con cualquier componente externo} que hace de montura para dos núcleos emuladoras y un microprocesador (\textit{que puede estar en una núcleo o en una placa personalizada}). Las núcleos emuladoras tienen un código fijo que puede recibir señales del microprocesador a testear y que dadas las señales que reciben trataran de simular una batería de periféricos para los que la librería fue planteada. \par
Las señales que las emuladoras reciban servirán para indicar que periféricos se quieren simular y para imitar las señales que reciben los periféricos seleccionados. Estos periféricos incluyen: Todos los protocolos de comunicación, un motor lineal con su encoder, sensores de temperatura y presión, válvulas, pwm, y otras núcleos con sus propios procesos. \par
En cuanto a el micro a testear, esta pensado para recibir el código a probar desde el bootloader, que deberá incluir todas las comunicaciones necesarias para el debugging y las señales para activar las núcleos emuladoras en los periféricos que se deseen probar en susodicho código. La SHUTP se diseño para poder ser usada tanto en La ATP como fuera de esta, pero para el caso de la ATP esta tendrá montada una núcleo en la conexión de la micro a testear y todas las comunicaciones con la OrangePi se harán a partir del bus PCAN\footnote{PCAN es un protocolo de comunicacion productor/consumidor donde los primeros responden a las peticiones del segundo.}.
\par \vspace{0.3cm}

Para la OrangePi se prepararon múltiples Gits en GitHub que contienen la librería, los códigos a probar, la batería de tests que estos deben pasar, y el propio código de la OrangePi. \par
El código de la OrangePi es un servidor de Python hecho en Flask que esta diseñado para revisar los Gits que contienen los códigos a probar en busca de cambios y pull-requests usando los WebHooks de GitHub. Una vez GitHub de un aviso de cambio en uno de estos Gits (o alternativamente un cliente se conecte al servidor OrangePi) el servidor creara un hilo de ejecución que descargara la rama a probar y ejecutara todos los tests comunicándose con la SHUTP, luego terminara su ejecución y devolverá la información al servidor OrangePi que dependiendo de si las pruebas dieron positivo o no aceptara el PR o dará información sobre los errores que sucedieron. En el caso de múltiples peticiones simplemente usa una cola FIFO\footnote{cola First In First Out, el primero que entra es el primero que sale, el tipo de cola mas común en el día a día}. \par
Los tests que deben pasar el código de la placa son comprobaciones que indican el orden en el que deben de suceder las cosas y comprueban a través del bus Ethernet si el resultado ha sido el esperado. Estos indican que código se ha de subir a la SHUTP, que peticiones se le han de mandar a dicho código a través de Ethernet, y dado estas peticiones que resultados le debe devolver la núcleo y en que margen de tiempo. Estos tests están hechos en Python y son ejecutados en orden alfabético y de forma atómica. Si un test falla las pruebas continúan y simplemente se registra el fallo. Si era una prueba de cliente, también se le da la opción de interrumpir los tests y se le avisa nada mas se da el fallo. \par
El código a probar debe ser uno que usando la ST-LIB emule una ejecución de un caso de uso real, este adaptado para activar las núcleos emuladoras de la SHUTP en los modos deseados para hacer las pruebas, y comunique continuamente los datos necesarios para que los tests puedan juzgar si la ejecución funciona como se espera y los resultados son los correctos. Este sera el código que se suba junto a la librería a través del Bootloader. La separación de test y código a testear es por comodidad. En lugar de hacer un gran test con múltiples datos para cada código, se separa en varios tests que comprueben distintas funcionalidades de forma modular, haciendo mas fácil añadir nuevos tests en el futuro sin afectar a los ya existentes.
\par \vspace{0.3cm}

TODO


\bibliographystyle{IEEEtran}
\bibliography{biblio}



\end{document}